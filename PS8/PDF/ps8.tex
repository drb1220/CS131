\documentclass{article}
\usepackage[utf8]{inputenc}

\title{Problem Set 8}
\author{David Bunger}
\date{November 2022}

\usepackage{graphicx}
\usepackage{amsmath, amssymb}
\usepackage{parskip}

\begin{document}

\maketitle

\section*{Problem 1}
Theorem:\\
For all integers $a$ and $b$ and for all natural numbers $n$, $a-b$ divides $a^n-b^n$.\\
Proof:\\
Base Case: $n=0$\\
$(a-b)|(a^0-b^0)$\\
$(a-b)|0$\\
Inductive Step:\\
Assume that $(a-b)|(a^k-b^k)$ for all natural numbers k, meaning there exists some integer $c_k$ such that $(a-b)c_k=(a^k-b^k)$. We must show that $(a-b)|(a^{k+1}-b^{k+1})$.
$a^{k+1}-b^{k+1}$\\
$a^{k+1}-b^{k+1}+ab^k-ab^k$\\
$a^{k}a-ab^k-b^{k}b+ab^k$\\
$a(a^k-b^k)-b^k(-a+b)$\\
$a(a^k-b^k)+b^k(a-b)$\\
$a(a-b)(c_k)+b^k(a-b)$ by Inductive Hypothesis\\
$(a-b)(a(c_k)+b^k)$\\
Since $(a-b)$ is a factor of $(a-b)(a(c_k)+b^k)$, $(a-b)|(a-b)(a(c_k)+b^k)$, and therefore, $(a-b)|(a^{k+1}-b^{k+1})$.
\clearpage
\section*{Problem 2}
\subsection*{a)}
$f(x)=\overline{x}+x$\\
$h(x)=x$\\
$g(x)=\overline{x}$\\
$r(x)=\overline{x}x$
\subsection*{b)}
Let $s(k)$ be the size of a function $f$ on $k$ variables, where k is a natural number. Assume that a function $f$ on $k$ variables, where k is a natural number, can be represented by a formula of size less than or equal to $3(2^k)-4$, or $s(k)\leq 3(2^k)-4$.  We are trying to prove that $s(k+1) \leq 3(2^{k+1})-4$. When a variable is added to a function, the number of rows in the truth table doubles. Each of these two halves can be looked at as their own truth table with one fewer variable, meaning $s(k)\leq 3(2^k)-4$ by the inductive hypothesis. When combining these two formulas, they can simply be connected with either a conjunction or a disjunction(depending on how the original formula was constructed), making the total size, or $s(k+1)$, equal to $2s(k)+1$. Now we have simply have to prove that $2s(k)+1\leq 3(2^{k+1})-4$.\\
$2s(k)+1\leq2(3(2^k)-4)+1$\\
$=6(2^k)-7$\\
$=3(2^{k+1})-7$\\
$<3(2^{k+1})-4$
\clearpage
\section*{Problem 3}
Theorem:\\
For every $n\geq 2$, $f_n\geq (1.5)^{n-2}$\\
Proof:\\
Base Cases:\\
$n=2$: We have $f_2=1$, and $(1.5)^0=1$. Since $1\geq 1$, the first base case holds.\\
$n=3$: We have $f_3=2$, and $(1.5)^1=1.5$. Since $2\geq 1.5$, the second base case also holds.\\
Inductive Step:\\
Assume that $f_j \geq (1.5)^{j-2}$ holds for all $j$ from 2 to $k$, where $k\geq 3$. We must show this holds for $k+1$, or that 
$f_{k+1}\geq (1.5)^{k-1}$\\
Based on our Inductive Hypothesis, we know that\\
$f_k\geq(1.5)^{k-2}$\\
$f_{k-1}\geq(1.5)^{k-3}$\\
This gives us\\
$f_{k+1}=f_k+f_{k-1}$\\
$\geq (1.5)^{k-2}+(1.5)^{k-3}$\\
The statement $x^a < x^{a-1}+x^{a-2}$ is true so long as $x$ is less than a specific number, being the golden ratio, or $\frac{1+\sqrt{5}}{2}$. When x is equivalent to the golden ratio, $x^a = x^{a-1}+x^{a-2}$. Since $1.5$ is less than the golden ratio, we can conclude that\\
$\geq(1.5)^{k-1}$\\
Since we have shown that the equation holds for both the base case and the inductive step, we have successfully proven that $f_n\geq (1.5)^{n-2}$.
\clearpage
\section*{Problem 4}
Theorem:\\
For all positive integers $n$, $a_n=3\cdot 2^{n-1}+2(-1)^n$\\
Proof:\\
Base Cases:\\
$n=1$: We have $a_1=1$, and $3\cdot 2^0+2(-1)^1=3-2=1$, so the first base case holds.\\
$n=2$: We have $a_2=8$, and $3\cdot 2^1+2(-1)^2=6+2=8$, so the second base case also holds.\\
Inductive Step:\\
Assume that $a_j=3\cdot 2^{j-1}+2(-1)^j$ holds for all $j$ from 1 to $k$, where $k\geq 2$. We must show that this holds for $k+1$, or that $a_{k+1}=3\cdot 2^k+2(-1)^{k+1}$\\
Based on our Inductive Hypothesis, we know that\\
$a_k=3\cdot 2^{k-1}+2(-1)^k$\\
$a_{k-1}=3\cdot 2^{k-2}+2(-1)^{k-1}$\\
This gives us\\
$a_{k+1}=a_k+2a_{k-1}$\\
$=(3\cdot 2^{k-1}+2(-1)^k)+2(3\cdot 2^{k-2}+2(-1)^{k-1})$\\
$=3\cdot 2^{k-1}+2(-1)^k+3\cdot 2^{k-1}+4(-1)^{k-1}$\\
$=6\cdot 2^{k-1}+2(-1)^k+4(-1)^{k-1}$\\
$=3\cdot 2^{k}+2(-1)^k+4(-1)^{k-1}$\\
Because $(-1)^k=-(-1)^{k-1}$,\\
$=3\cdot 2^{k}+-2(-1)^{k-1}+4(-1)^{k-1}$\\
$=3\cdot 2^{k}+2(-1)^{k-1}$\\
Since $(-1)^{k-1}=(-1)^{k+1},$\\
$=3\cdot 2^{k}+2(-1)^{k+1}$\\
Since we've shown the theorem holds for both base cases and the inductive step, we have successfully proven the theorem.\\
We needed to use strong induction in this case because the equation contains $a_{k-1}$. With weak induction, we can only assume that $a_k$ is true, so this wouldn't work. However, with strong induction, we can assume every integer $j$ from 0 to $k$, which includes $k-1$, so we can assume the theorem holds for $a_{k-1}$.
\end{document}
